\chapter{Voraussetzungen und verwandte Arbeiten}
\label{cha:requirements_and_related_work}

%In this section, you should point out which requirements a good solution to the problem addressed in your thesis should fulfill.
%It makes sense to state functional as well as non-functional requirements.

%When it comes to the related work, you should then point out to which extent the existing works / proposed solutions already fulfill the requirements you introduced previously.

\section{Kontextbegriff}
Kontext ist zwar ein Begriff, unter dem sich die meisten Menschen intuitiv etwas vorstellen können, ihn zu erläutern oder gar korrekt und vollständig zu definieren, fällt allerdings um ein Vielfaches schwerer \cite{dey_understanding_2001}. Auch herrschen in verschiedenen Fachbereichen jeder davon mit anderen Sachverhalten und Problemen, welche die jeweiligen Autoren versuchen zu lösen, abweichende Auffassungen darüber, welche Ansprüche eine Definition erfüllen muss. Das erschwert eine allumfassende, konkrete Bestimmung zusätzlich \cite{hutchison_understanding_2005}. Ziel der Arbeit ist also nicht, Kontext abschließend zu definieren. Dies ist in Anbetracht der vielen verschiedenen Ansätze und dem fehlenden Konsens in der Literatur darüber, wie solch eine Definition aussehen soll \cite{alegre_engineering_2016, wei_liu_survey_2011} nicht zufriedenstellend möglich. Die Menge an Anwendungsfällen und damit auch die zu beachtenden Gesichtspunkte, die man für eine allen Ansprüchen genügende Definition benötigt, sind dafür zu umfangreich. Stattdessen wird versucht, mithilfe relevanter Darlegungen anderer Autoren die historische Entwicklung des Kontextbegriffs für den Bereich der Informatik im Allgemeinen und der Zugriffskontrolle im Speziellen darzustellen. Dies soll ein Verständnis dafür schaffen, auf welchen Grundlagen, Ansätzen und Ideen der Kontextbegriff dieser Arbeit entstanden und aufgebaut ist. Dies ermöglicht dem Leser eine Einordnung davon, wie Kontext und Kontextsensitivität verwendet werden und er kann einen Abgleich mit seiner eigenen Interpretation der Begrifflichkeiten durchführen.
\subsection{Historische Entwicklung}
Die Definition von Kontext ist auch in der Literatur seit jeher ein Thema. Eine der frühesten von Schilit et al. \cite{schilit_context-aware_1994} bestimmt als drei Hauptaspekte, an welchem Ort, in Gegenwart welcher anderen Personen und in der Nähe welcher Ressourcen sich ein Nutzer befindet. Des Weiteren beinhaltet nach \cite{schilit_context-aware_1994} Kontext Attribute wie Beleuchtung, Lautstärke, den Grad der Netzwerkverbindung, Kommunikationskosten, und die soziale Situation.\\ Eine der seit ihrer Entstehung in 2001 am häufigsten zitierten Definitionen stammt von Dey \cite{dey_understanding_2001}. Laut dieser ist Kontext "jede Information, die genutzt werden kann, um die Situation einer Entität zu charakterisieren. Eine Entität ist eine Person, ein Objekt oder ein Ort mit Relevanz für die Interaktion zwischen Nutzer und Anwendung. Das schließt auch Nutzer und Anwendung selbst mit ein". Sie wird allgemein hin von den vielen anderen Autoren als Quasikonsens akzeptiert \cite{aguilar_cameonto_2018,alegre_engineering_2016,wei_liu_survey_2011} oder sogar als Ausgangspunkt für ihre eigene Definition genutzt \cite{kayes_icaf_2012, kokinov_operational_2007}.\\ Kaltz et al. \cite{wolfgang_kaltz_context-aware_2005} verstehen Kontext als einen Kontextraum, also eine Kombination aus Kontextparametern, Elementen einer domänenspezifischen Ontologie und Dienstleistungsbeschreibungen in Form von $C = \{U;P;L;T;D;I;S\}$. Dabei ist $U$ das Set aus Nutzern und den dazugehörigen Rollen, $P$ die Prozesse und Aufgaben,  $L$ der Ort,  $T$ der Zeitfaktor,  $D$ beschreibt das Gerät, $I$ die verfügbaren Informationen und $S$ die verfügbaren Dienstleistungen. Ein spezifischer Kontext ist somit ein Punkt in diesem Raum.\\ Einen anderen Ansatz wählen Bazire und Brézillon. Sie haben 150 Kontextdefinitionen analysiert und sind dabei zu der Erkenntnis gekommen, das Kontext wie eine Begrenzung fungiert, welche das Verhalten eines Systems, Nutzers oder Computers in einer bestimmten Tätigkeit beeinflussen. Allerdings herrscht ihrer Ansicht nach kein Konsens  darüber, ob Kontext extern oder intern, ein Set aus Informationen oder Abläufen, statisch oder dynamisch ist  \cite{hutchison_understanding_2005}.\\ Im Bezug auf kontextsensitive Zugriffskontrolle sind die Arbeiten von Kayes et al. \cite{kayes_icaf_2012,kayes_survey_2020,kayes_ontological_2015} erwähnenswert. In \cite{kayes_icaf_2012} definieren diese Kontextinformationen, in Bezug auf Zugriffskontrollentscheidungen, als relevante Informationen über den Zustand einer Entität (Nutzer, Ressource, Ressourcenbesitzer) und deren Umgebung oder die Beziehung zwischen Entitäten.\\
\section{Kontextkategorisierung}
Genauso relevant für diese Arbeit wie eine Übersicht gängiger Definitionen ist ein Überblick darüber, wie Kontext kategorisiert werden kann. Nach dem Kontextverständnis in der Literatur wird also im Folgenden die Kategorisierung von Kontext näher betrachtet. Dazu erfolgt in diesem Kapitel eine Vorstellung verschiedener ausgewählter Unterteilungsansätze. Auf diese wird sich auch bei der Erstellung der Taxonomie dieser Arbeit in \ref{cha:design} bezogen.
\subsection{Übersicht  relevanter bereits vorgeschlagener Kategorien} 
\label{cha:3_2_1}
Auch bei der Kategorisierung von Kontext gibt es richtungsweisende Vorschläge:\\\\ Abowd et al. \cite{abowd_towards_1999} haben einen der wegweisenden Mechanismen zur Definition von Typen von Kontext vorgeschlagen. Sie identifizierten Ort, Zeit, Identität und Aktivität als primäre Kontexttypen. Weiterhin wird sekundärer Kontext als etwas definiert, das durch Nutzung von Primärkontext erschlossen werden kann.\\\\ Schilit et al. \cite{schilit_context-aware_1994} kategorisieren Kontext basierend auf drei Fragen, die genutzt werden können, um den Kontext zu bestimmen, in drei Kategorien:
\begin{enumerate}
\item{Informationen, die sich auf einen Ort beziehen, beispielsweise GPS-Koordinaten, Bezeichnungen von Institutionen oder Gebäuden (ein Café, ein Krankenhaus, eine Universität), spezifische Namen (z. B.: Technische Universität Dresden) spezifischen Adressen(z. B.: APB Nöthnitzer Str. 46) oder Nutzerpräferenzen (z. B. das Lieblingsrestaurant eines Nutzers) }
\item{Informationen über andere Personen, die in der Nähe aufhalten}
\item{Informationen darüber, welche Ressourcen (Maschinen, technische Geräte, Betriebsmittel) sich im direkten Umfeld eines Nutzers befinden}
\end{enumerate}
Henricksen \cite{henricksen2003framework} ordnet Kontext basierend auf der betrieblichen Kategorisierungstechnik in 4 verschiedene Kategorien:
\begin{enumerate}
\item {Messbar: Informationen, die aus unmittelbar messbaren Werten bestehen. Diese ändern sich oft oder gar kontinuierlich. }
\item {Statisch: Informationen, die sich während der Lebenszeit eines Systems gleich bleiben.}
\item {Profiliert: Informationen, die sich selten ändern.}
\item {Abgeleitet: Informationen, die unter Verwendung anderer Daten gewonnen wurden. }
\end{enumerate}
Van Bunningen et al. \cite{van2005context} ordnen Kategorisierungsversuche in zwei übergeordnete Gruppen: Betrieblich und Konzeptionell. 
\begin{enumerate}
\item{betriebliche Kategorisierung: Einordnung anhand dessen, wie der Kontext akquiriert, modelliert und behandelt wird.}
\item{konzeptionelle Kategorisierung: Einordnung anhand der Bedeutung des Kontextes und der konzeptionellen Beziehungen}
\end{enumerate}
Chong et al. \cite{chong_context-aware_nodate} schlagen Historie als Kontextkategorie vor. Dabei wird die zeitliche Entwicklung einer bestimmten Messgröße als Kontext definiert. Das erlaubt die Festlegung eines Normalzustandes für dieser Größe. Das ermöglicht laut den Autoren unter Umständen eine Vorhersage darüber, welche Werte die Messgrößen zukünftig annehmen könnten.
