\chapter{Voraussetzungen und verwandte Arbeiten}
\label{cha:requirements_and_related_work}

%In this section, you should point out which requirements a good solution to the problem addressed in your thesis should fulfill.
%It makes sense to state functional as well as non-functional requirements.

%When it comes to the related work, you should then point out to which extent the existing works / proposed solutions already fulfill the requirements you introduced previously.

%\subsection{Angreifermodell}

\subsection{Kontextsensitivität}
%“A system is context-aware if it uses context to provide relevant information and/or services to the user, where relevancy depends on the user’s task.” 

Ein System ist kontextsensitiv wenn es Kontext verwendet um dem Nutzer mit für ihn relevanten Informationen und/oder Dienstleistungen zu versorgen. Die Relevanz hängt dabei von der Aufgabe des Nutzers ab.
\subsection{Situation}
%“As defined by Endsley [76], the situation comprises the perception of the elements in the environment within a volume of time and space, the comprehension of their meaning, and the projection of their status in the near future.”
\subsection{Umgebung}

\subsection{Netzwerk-Sicherheits-Monitoring}
\subsection{Policy}
\subsection{IDS}
\subsubsection{Intrusion}
\subsubsection{Intrusion Detection}


