%\begin{abstract}
%   The abstract serves as a summary of the work. It should concisely (around half a page) answer the following questions:
%   \begin{itemize}
%      \item What is the problem tackled in this thesis?
Die Aufrechterhaltung und Verbesserung der Netzwerksicherheit, sowie die Anpassungsfähigkeit der dazu eingesetzten Schutzmechanismen an immer neue Gegbenheiten, sind seit dem bestehen von IT-Netzwerken von ein ständiges Problem.
%      \item Why is this problem relevant?
Netzwerke werden immer weitreichender und komplexer. Sowohl die in das Netzwerk zu integrierenden und zu verwaltenden Systeme, als auch deren Verwendungsmöglichkeiten werden immer vielfältiger. Dadurch unterliegen auch das damit verbundene immense Netzwerkverkehrsaufkommen und die dadurch möglich werdenden Angriffe einem stetigen Wandel. Deshalb müssen auch bisher erdachte Schutzkonzepte immer wieder angepasst und die dafür verwendete Software stetig verbessert werden. Eins der zentralen Konzepte um diese Aufgaben zu bewältigen ist Kontextsensitivität.
%      \item What contributions does this thesis contain with regards to the problem?
Diese Arbeit gibt einen Überblick über den aktuellen Stand im Bereich der Kategorisierung von Kontext. Präsentiert eine Variante diese unter dem Gesichtpunkt des Netzwerk-Sicherheits-Monitoring in Verbindung zu bringen. Zeigt auf wo und wie diese Informationen gesammelt werden können. Betrachtet werden dabei konkrete Leistungskriterien mit Hilfe ausgewählter Anwendungsfälle. Darauf aufbauend erfolgt abschließend eine Bewertung der vorgestellten Kontextkategorien im Bezug auf den Nutzen zur Verbesserung eines IDS. Das soll den Lesenden dazu befähigen, selbst Anpassungen an der Arbeitsweise des eigenen Systems vorzunehmen und so die Netzwerksicherheit nachhaltig zu erhöhen. 
%      \item Which scientific method(s) was/were used and what were their results?

%   \end{itemize}
%\end{abstract}
