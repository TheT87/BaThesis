
\chapter{Hintergrund}%
\label{cha:background}

The background chapter aims to give a more detailed introduction into the context of your work and provides additional information that enables the target audience to understand the argumentations and motivations that you explain in the following chapters.
This chapter is typically read on demand, if in the following chapters, the reader encounters a term or argumentation that is not immediately clear from the corresponding chapter alone.

\section{Definition eines Kontextbegriffs im Rahmen der Zugriffskontrolle}

\subsection{Historische Entwicklung des Kontextbegriffs}

%“where you are, who you are with, and what resources are nearby (see Figure 2.).” 
Eine der frühesten Definitionen von Kontext \cite{schilit_context-aware_1994} bestimmt als 3 Hauptaspekte von Kontext an welchem Ort, mit wem und in der Nähe welcher Ressourcen man sich befindet.
%“lighting, noise level, network connectivity, communication costs, communication bandwidth, and even the social situation; e.g., whether you are with your manager or with a co-worker.”
Desweiteren beinhaltet laut \cite{schilit_context-aware_1994} Kontext Attribute wie Beleuchtung, Lautstärke, den Grad der Netzwerkverbindung, Kommunikationskosten, Kommunikationsbandbreite und die soziale Situation.
%“Context is any information that can be used to characterise the situation of an entity. An entity is a person, place, or object that is considered relevant to the interaction between a user and an application, including the user and applications themselves.” 
Nach Dey \cite{dey_understanding_2001}, eine der am meist akzeptiertesten, wenn nicht sogar die akzeptierteste, Definition, ist "Kontext jede Information die genutzt werden kann um die Situation einer Entität zu charakterisieren. Eine Entität ist eine Person, ein Objekt oder ein Ort mit Relevanz für die Interaktion zwischen Nutzer und Anwendung. Das schließt auch Nutzer und Anwendung selbst mit ein".

%“Context is thus determined by a set of contextual factors and a current perspective on these factors, or,as termed in (Brezillon, 2004), the contextual knowledge.This knowledge results from the context sensing and user behavior; an integrated ontology; a structuring of the contextual factors, resulting from the context modeling.” 

\cite{wolfgang_kaltz_context-aware_2005}

%“Our analysis of definitions of context collected on the Web shows that context could be analyzed through six essential components. The context acts like a set of constraints that influence the behavior of a system (a user or a computer) embedded in a given task. We discuss the nature and structure of context in the definitions. There is no consensus about the following questions: Is context external or internal? Is context a set of information or processes? Is context static or dynamic? Is context a simple set of phenomenon or an organized network?” 

%“context is the set of circumstances that frames an event or an object.” 

%“Context is any information that can be used to characterize the situation of an entity [7]. Elements for the description of this context information fall into five categories: individuality, activity, location, time, and relations.” 

%“At the time of this writing, there is still no standard way of defining the term of context. As far, most researchers even accept a more general definition made by Dey [7]” 

%“Context information’ used in an access control decision is defined as any relevant information about the state of a relevant entity (user, resource, resource owner and their environments) or the state of a relevant relationship between the entities.” 
%based on Dey

%“Although “context” is a term that most people tacitly understand, they find it difficult to elucidate [3]. Many multidisciplinary areas use context to enhance their possibilities. Each area understands the notion as a reflection of its own concerns, making it difficult to define [4]. In the literature, several definitions can be found [3] [4] [5] [6] [7] [8] [9]. A detailed comparison between the differences and similarities of these is out the scope of this survey. Nevertheless, it has to be acknowledged that there is no consensus on the definition of context. Also, we highlight that Dey’s [10] is the most acknowledged one, considering it as “any information that can be used to characterize the situation of an entity”, where “an entity can be a person, place, or object that is considered relevant to the interaction between a user and an application, including the user and applications themselves”.” 

%“Context: Dey expresses that for the computation [15]: ‘‘Context is any information that can be used to characterize the situation of an entity. An entity is a person, place or object that is considered relevant to the interaction between a user and an application.”” 

\subsection{Darlegung/Erläuterung meiner Kontextdefinition}

\section{Kontextkategorisierung}

		\subsection{ Analyse bereits vorgeschlagener Kategorien} 
		\subsection{ Erstellung einer Kontexttaxonomie }

