\chapter{Zusammenfassung und Ausblick}%
\label{cha:conclusion}

%In this chapter, you summarize the conclusions that can be drawn from your thesis with regards to solving the problem explained in the introduction section.
%Furthermore, you should concisely explain further experiments or design options that may be interesting to pursue in future work.

\section{Offene Forschungsfragen}
\subsection{•} 
\subsection{Dateilose Prozesse}
Die für die Implementation verwendete Software erlaubt es lediglich Informationen von Prozessen die in ihrem Lebenszyklus Daten die Festplatte schreiben zu akquirieren. Netzwerkverkehr von Schadsoftware die ausschließlich im Arbeitsspeicher eines Systems arbeitet kann so nicht durch die in dieser Arbeit vorgestellten Methoden mit zusätzlichen Kontextinformationen kombiniert werden. Prozesse die nur im Arbeitsspeicher existieren wurden deshalb auch nicht betrachtet sind aber ein zunehmendes Problem im Bereich der IT-Sicherheit.
%TODO Citation needed  
%\subsection{Nutzdaten von Verbindungen}
%Nutzdaten sind sowohl ein 