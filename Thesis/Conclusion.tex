\chapter{Zusammenfassung und Ausblick}%
\label{cha:conclusion}

%In this chapter, you summarize the conclusions that can be drawn from your thesis with regards to solving the problem explained in the introduction section.
%Furthermore, you should concisely explain further experiments or design options that may be interesting to pursue in future work.

Es gibt eine große Menge an möglichen Ansätzen Kontext einzuteilen. Einige davon wurden in dieser Arbeit vorgestellt, zwei selbst in Kapitel \ref{cha:design} konstruiert, in Kapitel \ref{cha:implementation} exemplarisch getestet und anschließend in Kapitel \ref{cha:evaluation} ausgewertet. Es hat sich dabei gezeigt, dass Kategorien bis auf wenige Ausnahmen erst in Kombination mit anderen Kategorien ihr volles Potenzial entfalten können. Auch hat selten eine Kategorie allein alle Leistungsmetriken gleichzeitig abdecken können. Zeek hat sich dabei aufgrund seiner vielfältigen Möglichkeiten als ein sehr gutes Werkzeug zum Testen der Kategorien herausgestellt. Die Möglichkeiten, welche die gewählte Implementation bietet, wenn erst einmal verstanden wurde, wie es zu nutzen sind, fast unbegrenzt. Zeek-Agent hingegen hat sich als weniger mächtig in seinen Möglichkeiten als zunächst vermutet herausgestellt. Positiv zu erwähnen ist die hervorragende Einbindung in Zeek. Negativ ist allerdings die im Vergleich zu anderen solchen Werkzeugen geringe Anzahl an abfragbaren Informationen. Hier bietet es sich also an, eventuell zusätzlich andere, in Kapitel \ref{cha:design} vorgestellte Mittel zu verwenden. Zusammengefasst lässt sich aber sagen, dass mehr Kontextinformationen in den allermeisten Fällen die Leistung verbessern.\newline
%\section{Offene Forschungsfragen}
\subsection{Dateilose Prozesse}
Ein in dieser Arbeit nicht betrachteter Aspekt sind Prozesse, die in ihrem Lebenszyklus keine Daten auf die Festplatte schreiben oder anderweitig nachweisbar Spuren hinterlassen. Es wurde sich nur mit Vorgängen beschäftigt, die nachweislich einlesbare Daten erzeugen oder für die Gewährleistung ihrer Funktionalität zumindest benötigen. Netzwerkverkehr von Schadsoftware, die beispielsweise ausschließlich im Arbeitsspeicher eines Systems arbeitet, kann nicht mit zusätzlichen Kontextinformationen kombiniert werden. Hier wäre eine zukünftige Betrachtung der Erkennungsmöglichkeiten also dringend notwendig, da gerade diese Art von Software ein zunehmendes Problem nicht nur im Bereich der IT-Sicherheit darstellt und es nach Wissensstand des Autors bisher noch keine ausgereiften Ansätze zur Bekämpfung gibt.
%TODO Citation needed  
%\subsection{Nutzdaten von Verbindungen}
%Nutzdaten sind sowohl ein 