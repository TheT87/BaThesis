\chapter{Evaluation}%
\label{cha:evaluation}

%This chapter is usually expected to present which experiments you did as part of your thesis, what results came out of them and what these results tell us about to which extent your design improves the state of the art with regards to the requirements specified in chapter
%As a rough outline, this chapter should address the following questions:

%\section{Auswertung der Testergebnisse}

\section{Vergleich der Kontextkategorien} 
%\subsection{Theoretische Mächtigkeit einzelner Kategorien} 
\subsection{Verfügbarkeit der Informationen}
Die Informationen die ich in den definierten Kategorien als gegeben vorausgesetzt habe sind in der Praxis unterschiedlich schwer zu akquirieren. Offensichtlich ist es um ein Vielfaches einfacher korrekt die aktuelle Uhrzeit zu ermitteln als den Grund dafür das ein Nutzer eine Aktion ausführt zu bestimmen. Die nachfolgende Grafik dient dazu den meiner Einschätzung nach benötigten Aufwand zur Beschaffung einer Information zu veranschaulichen.
%TODO Grafik
\subsection{Qualität der Informationen}

\includegraphics[width=13.5cm,height=24.1cm]{test}

\section{Erhöhung der Netzwerksicherheit}
\subsection{Nicht-kontextsensitive Signaturen}
\subsection{Kontextsensitive Signaturen} 
