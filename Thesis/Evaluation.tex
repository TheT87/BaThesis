\chapter{Evaluation}%
\label{cha:evaluation}

This chapter is usually expected to present which experiments you did as part of your thesis, what results came out of them and what these results tell us about to which extent your design improves the state of the art with regards to the requirements specified in chapter~\ref{cha:requirements_and_related_work}.
As a rough outline, this chapter should address the following questions:
\begin{itemize}
   \item Which questions did you want to answer or which hypothesises did you want to test with the experiments?
   \item Which metrics did you measure and how does their value relate to the questions you want to answer?
   \item Which system parameters exist that may have an influence on the value on the metric? Which ones did you vary in your experiments? Intuitively, what are your expectations with regards to the relationship between the system parameters and the metrics?
   \item How did the experiments that you did actually looked like, or how did you actually measure the chosen metrics?
   \item What are the actual values that you measured in your experiments? Do they match your intuition about the relationship between the system parameters and the metrics?
\end{itemize}

%\section{Auswertung der Testergebnisse}
\section{ Vergleich der IDS-Alerts mit Datensatzlabeln}
\subsection{ Baselineauswertung } 
\subsection{ Vergleich der Performance gemäß der in 1.1 festgelegten Kriterien zwischen non-kontextsenitiven und kontextsenitiven Signaturen} 
\subsection{ Vergleich der unterschiedlichen getesteten Kontextkategorie(kombinationen)} 

\section{ “Bewertung” der theoretischen Mächtigkeit einzelner Kategorien} 
\section{ Beurteilung/Einschätzung der tatsächlichen Verfügbarkeit von Kontext im Netzwerk }
\section{ Urteil/""Ranking"" der Kontextarten hinsichtlich der Erhöhung der Netzwerksicherheit}
