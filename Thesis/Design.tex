\chapter{Design}%
\label{cha:design}

%In this chapter, you should present your solution in detail but at a conceptual level. This means that you explain the overall design including your motivation for this design but you do not provide details on the actual implementation of your design (e.g., in which programming language you wrote it, how the software is structured and so on). This means that you should also point out the aspects where you had different design options and in which points they differ. A good approach to write this chapter is to make yourself aware of the different aspects and design problems that need to be addressed in your design. To do so, you can then proceed repeatedly in three steps:
%\begin{enumerate}
 %  \item Explain a design problem that needs to addressed by the solution (e.g. to enable anonymous communication over the internet, participants need to be able to send messages to each other without revealing identifying to the corresponding receiver).
  % \item Discussion of design choices (e.g. Mix Networks, DC-Networks, etc.) with regards to the requirements from the previous chapter and identification of the most promising choices.
%\end{enumerate}
%After the second step, you start the next iteration by identifying design problems that arise when you want to use the most promising design choice. For example, if Mix networks turn out to be the most promising approach for your requirements, you then need to address the question how the mix network should be designed (e.g. how are mix nodes chosen by the users of the anonymization network? How do mix nodes process messages?). Once you identified the most promising solutions to that, you can then start the next iteration and so on until there are no more open design questions that you are aware of.

\section{Umwandlung der Taxonomie in IDS-Signaturen} 

%What is important for the reader is to understand that the contextual awareness of machines is from a radically different nature than the one of humans. Also, that computational systems are good at gathering and aggregating data, but humans are still better at recognizing contexts and determining what action is appropriate in a certain situation [2]. On the other hand, positivism looks at context as a representational problem, considering it as a “form of information, delineable, stable and separable from activity” [5]. The definitions made in the context-aware field, naturally adopt this point of view. For instance, Dey’s definition [10] allows designers to use the concept for determining why a situation occurs and use this to encode some action in the application [26], making the concept operational in terms of the actors and information sources [17]. Nevertheless, since the definition inherently has a positivist view, the potential of C-AS remains limited to the context that developers are able to encode and foresee. \cite{alegre_engineering_2016}
Um die im Abschnitt Kontexttaxionomie festgelegten Kategorien in Signaturen für ein Intrusion Detection System umzuwandeln zu können gilt es gewisse Dinge zu beachten:\\\\ Die Kontextsensitivität eines Computers unterscheidet sich drastisch von der eines Menschen. Rechensysteme sind sehr gut darin Daten zu erfassen und zu sammeln, aber Menschen sind immer-noch nötig um verschiedene Kontexte zu erkennen und zu entscheiden welches Handeln in einer bestimmten Situation angemessen ist \cite{dey_understanding_2001}.\\
Der limitierende Faktor des Potenzials eines kontextsensitiven Systems ist das Maß an Kontext das ein Entwickler vorhersehen und kodieren kann.\\
%The list of unforeseen or undetectable contexts can be endless.Summarizing, if developers can not determine all that can be affected by an action, it will be very difficult to write a closed and comprehensive set of actions to take in those cases.
Es ist aber weder beim Design noch später bei der Implementierung unmöglich alle Zusammenhänge vorherzusehen.\\
Dementsprechend schwer wird ist es ein in sich geschlossenes und allumfassendes Regelset festzulegen \cite{perera_context_2014}.\\\\
%“(A) Enumerate the set of contextual states that may exist;” 
%“(B) Know what information could accurately determine a contextual state within that set;” 
%“(C) State what appropriate action should be taken in a particular state.”
 Nach Greenberg et al. \cite{greenberg2001context} gibt 3 non-triviale Hauptaspekte die man beim Entwerfen eines kontextsensitiven Systems beachten sollte: 
%Mehr erklärung nötig?
\begin{enumerate}
\item{Spezifizieren aller möglichen Kontextzustände}
\item{Wissen welche Informationen einen konkreten Kontextzustand akkurat festlegen.}
\item{Welche Aktion im jeweiligen Zustand ausgeführt werden sollen.}
\end{enumerate}

\subsection{ Festlegung auf bestimmte IDS-Implementierungen}
Es steht eine Vielzahl an Software- und Hardwareimplementierungen zur Entscheidung von Zugriffskontrollentscheidungen zur  Verfügung.
Je nachdem welche Kriterien auf welche Art und Weise erfüllt werden müssen hängt vom spezifischen Anwendungsfall ab.
%TODO meine Kriterien spezifizieren + evtl kurze Erklärung zeek+suricata
\subsection{Analyse der Regel-Syntax eines IDS}
Wie arbeiten die einzelnen IDS? Wie einfach lassen sich die Zustände die auftretender Kontext annehmen kann in Regeln festlegen?
\subsection{ Abgleich zwischen Taxonomie und den in der Praxis zur Verfügung stehenden Informationen}
Wie gut lässt sich bestimmter Kontext in verfügbaren Informationen finden bzw. sind die verfügbaren Informationen in die Taxonomie einsortierbar?
\section{Anspruch an das IDS}
Um die Performance eines IDS verbessern zu können, muss man festlegen welche Kriterien die Leistung beeinflussen bzw. bestimmen.
Mell et al.\cite{mell2003overview} nennen in ihrer Übersicht verschiedene Charakteristiken zur quantitativen Bestimmung der Erkennungsgenauigkeit eines IDS und erläutern zusätzlich die Wechselwirkungen zwischen einzelnen Kriterien, die beim Vergleich verschiedener IDS-Lösungen beachtet werden sollten.
%Coverage: This measurement determines which attacks an IDS can detect under ideal conditions. 
\subsubsection{Abdeckung}
Gibt an welche Typen von Angriffen ein IDS unter idealen Bedingungen feststellen kann.
% Probability of False Alarms 
%This measurement determines the rate of false positives produced by an IDS in a given environment during a particular time frame. A false positive or false alarm is an alert caused by normal non- malicious background traffic
\subsubsection{Wahrscheinlichkeit falscher Alarme}
Gibt die Wahrscheinlichkeit das durch ein IDS ausgelöste Alarme durch gutartigen bzw. nicht-schädlichen Netzwerkverkehr verursacht wurden an.\\
\[Rate\;an\;falsch-Positiven\;Meldungen = \frac{Anzahl\;falscher\;Alarme}{Anzahl\;aller\;Alarme}\]
%Probability of Detection: 
%This measurement determines the rate of attacks detected correctly by an IDS in a given environment during a particular time frame
\subsubsection{Wahrscheinlichkeit einer Erkennung}
Gibt die Rate der durch das IDS korrekt erkannten Angriffe an.\\
\[Erkennungswarscheinlichkeit  = \frac{Anzahl\;korrekt\;erkannter\;Angriffe} {Anzahl\;aller\;Angriffe}\]
%Resistance to Attacks Directed at the IDS This measurement demonstrates how resistant an IDS is to an attacker's attempt to disrupt the correct operation of the IDS. Attacks against an IDS may take the form of: 
%1. Sending a large amount of non-attack traffic with volume exceeding the IDS’s processing capability. With too much traffic to process, an IDS may drop packets and be unable to detect attacks. 
%2. Sending to the IDS non-attack packets that are specially crafted to trigger many signatures within the IDS, thereby overwhelming the IDS’s human operator with false positives or crashing alert processing or display tools. 
%3. Sending to the IDS a large number of attack packets intended to distract the IDS’s human operator while the attacker instigates a real attack
\subsubsection{Resistenz}
Ein signatur-basiertes IDS bzw. der menschliche Administrator hinter dem System weisen Probleme auf die nicht direkt beim Umgang mit verarbeitetem Netzwerkverkehr, sondern schon bei der bewussten, unbewussten, oder erzwungenen Entscheidung welcher Netzwerkverkehr überhaupt in Frage kommt, entstehen:
\begin{enumerate}
\item{Ein zu große Menge an zu verarbeitendem Netzwerkverkehr die die Verarbeitungskapazität eines IDS übersteigt kann dazu führen das Netzwerkpakete verworfen und Angriffe nicht erkannt werden}
\item{Pakete die zwar nicht bösartig sind aber so konstruiert das sie möglichst viele IDS Signaturen auslösen, überfordern den Administrator oder stören eventuell sogar die Verarbeitung von Paketen generell. }
\item{Ein Angreifer könnte eine Vielzahl "harmloserer" aber trotzdem noch als schädlich zu deklarierende Pakete senden um einen größeren Angriff im Netzwerkverkehr zu verschleiern.}
\item{Pakete die möglicherweise vorhandene Fehler im IDS selbst ausnutzen.}
\end{enumerate}
%Ability to Correlate Events: This measurement demonstrates how well an IDS correlates attack events. These events may be gathered from IDSs, routers, firewalls, application logs, or a wide variety of other devices.
\subsubsection{Korrelation zwischen Einzelereignissen }
Demonstriert wie gut ein IDS eine Korrelation zwischen einzelnen Ereignissen, möglicherweise verschiedenen Ursprungs herzustellen. Die Ereignisse können dabei aus Routern, Firewalls, Anwendungen, dem IDS selbst oder einer großen Bandbreite anderer Quellen stammen.
%Ability to Detect Never Before Seen Attacks: This measurement demonstrates how well an IDS can detect attacks that have not occurred before. For commercial systems, it is generally not useful to take this measurement since their signature-based technology can only detect attacks that had occurred previously (with a few exceptions). However, research systems based on anomaly detection or specification-based approaches may be suitable for this type of measurement. Usually systems detecting attacks that had never been detected before produce more false positives than those that do not have this feature.
\subsubsection{Unbekannte Angriffe vorhersehen}
Gibt an wie gut ein IDS Angriffe erkennt die so noch nicht aufgetreten sind. Signatur-basierte IDS sind allgemein, mit wenigen Ausnahmen, nicht in der Lage solche Angriffe zu erkennen. Normalerweise erhöht die Fähigkeit eines Systems noch unbekannte Angriffe zu erkennen, im Vergleich zu Systemen die dies nicht versuchen, zusätzlich die Rate an falsch-positiven Meldungen.
%Ability to Identify an Attack: This measurement demonstrates how well an IDS can identify the attack that it has detected by labeling each attack with a common name or vulnerability name or by assigning the attack to a category.
\subsubsection{Identifizieren von Angriffen}
Wie gut ein IDS Angriffe die es erkennt, einen Namen oder einer Kategorie, beispielsweise ein CVE-Nummer, zuordnen kann.
%Ability to Determine Attack Success: This measurement demonstrates if the IDS can determine the success of attacks from remote sites that give the attacker higher- level privileges on the attacked system. In current network environments, many remote privilege-gaining attacks (or probes) fail and do not damage the system attacked. Many IDSs, however, do not distinguish the failed from the successful attacks. For the same attack, some IDSs can detect the evidence of damages (whether the attack has succeeded) and some IDSs detect only the signature of attack actions (with no indication whether the attack succeeded or not). The ability to determine attack success is essential for the analysis of the attack correlation and the attack scenario; it also greatly simplifies an analyst’s work by distinguishing between more important successful attacks and the usually less damaging failed attacks. Measuring this capability requires the information about failed attacks as well as successful attacks.
\subsubsection{Beurteilung  eines Angriffs}
Indikator dafür ob ein IDS den Erfolg und die Auswirkungen eines Angriffes korrekt beurteilen kann. In aktuellen Netzwerkumgebungen schlagen viele Angriffs(-vesuche) fehl. Die meisten IDS unterscheiden allerdings nicht zwischen erfolgreichen und fehlgeschlagenen Angriffen. Für den selben Angriff können manche IDS die Anzeichen dafür ob ein Angriff erfolgreich war erkennen, andere lediglich das ein Angriff stattgefunden hat, allerdings ohne feststellen zu können ob er erfolgreich war. Die Fähigkeit, den Grad zu dem ein Angriff auf das überwachte System erfolgreich war zu beurteilen ist essenziell. Eine Vorfilterung der Meldungen durch das IDS vereinfacht die Arbeit des Netzwerkadministrators bzw. Analysten stark, da so eine Analyse des Angriffsszenarios und der Korrelation einzelner Angriffe vereinfacht wird. Diese Fähigkeit bei einem gegebenen IDS messen zu können setzt das Wissen, darüber welche Angriffe erfolgreich sind und welche nicht, voraus.

% Other Measurements
% There are other measurements, such as ease of use, ease of maintenance, deployments issues, resource requirements, availability and quality of support etc. These measurements are not directly related to the IDS performance but may be more significant

\subsection{}
% Also, the probability of detection varies with the false positive rate, and an IDS can be configured or tuned to favor either the ability to detect attacks or to minimize false positives
%TODO weiter erklären
Die Wahrscheinlichkeit einen Angriff zu erkennen variiert mit der Rate an falsch-positiven Meldungen. Bei der Konfiguration eines IDS kann entweder die Falsch-positiv-Rate oder die Erkennungsrate optimiert werden.

%unterscheidung performance/securiry related metrics 
\cite{milenkoski_evaluating_2015}
In dieser Arbeit beschäftige ich mich ausschließlich mit sicherheitsrelevanten Leistungsmetriken. Dabei will hauptsächlich Abdeckung, Wahrscheinlichkeit einer Erkennung, Wahrscheinlichkeit falscher Alarme und der generellen Interpretierbarkeit der Meldungen durch einen menschlichen Nutzer betrachten. Zusätzlich will ich beleuchten inwiefern Kontext die Korrelation zwischen Einzelereignissen und das Identifizieren von Angriffen ermöglicht bzw. verbessert.
%One of the primary goals of this correlation is to identify staged penetration attacks. Currently, IDSs have only limited capabilities in this area.
%Überleitung von zu CAAC
\section{Kontextsensitivität}