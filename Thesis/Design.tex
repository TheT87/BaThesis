\chapter{Design}%
\label{cha:design}

%In this chapter, you should present your solution in detail but at a conceptual level. This means that you explain the overall design including your motivation for this design but you do not provide details on the actual implementation of your design (e.g., in which programming language you wrote it, how the software is structured and so on). This means that you should also point out the aspects where you had different design options and in which points they differ. A good approach to write this chapter is to make yourself aware of the different aspects and design problems that need to be addressed in your design. To do so, you can then proceed repeatedly in three steps:
%\begin{enumerate}
 %  \item Explain a design problem that needs to addressed by the solution (e.g. to enable anonymous communication over the internet, participants need to be able to send messages to each other without revealing identifying to the corresponding receiver).
  % \item Discussion of design choices (e.g. Mix Networks, DC-Networks, etc.) with regards to the requirements from the previous chapter and identification of the most promising choices.
%\end{enumerate}
%After the second step, you start the next iteration by identifying design problems that arise when you want to use the most promising design choice. For example, if Mix networks turn out to be the most promising approach for your requirements, you then need to address the question how the mix network should be designed (e.g. how are mix nodes chosen by the users of the anonymization network? How do mix nodes process messages?). Once you identified the most promising solutions to that, you can then start the next iteration and so on until there are no more open design questions that you are aware of.


\section{“Übersetzung”/Transfer der Taxonomie/Kategorien in IDS-Signaturen} 

%What is important for the reader is to understand that the contextual awareness of machines is from a radically different nature than the one of humans. Also, that computational systems are good at gathering and aggregating data, but humans are still better at recognizing contexts and determining what action is appropriate in a certain situation [2]. On the other hand, positivism looks at context as a representational problem, considering it as a “form of information, delineable, stable and separable from activity” [5]. The definitions made in the context-aware field, naturally adopt this point of view. For instance, Dey’s definition [10] allows designers to use the concept for determining why a situation occurs and use this to encode some action in the application [26], making the concept operational in terms of the actors and information sources [17]. Nevertheless, since the definition inherently has a positivist view, the potential of C-AS remains limited to the context that developers are able to encode and foresee. \cite{alegre_engineering_2016}
Um die im Abschnitt Kontexttaxionomie festgelegten Kategorien in Signaturen für ein Intrusion Detection System umzuwandeln zu können gilt es gewisse Dinge zu beachten: Die Kontextsensitivität eines Computers unterscheidet sich drastisch von der eines Menschen. Rechensysteme sind sehr gut darin Daten zu erfassen und zu sammeln, aber Menschen sind immer-noch nötig um verschiedene Kontexte zu erkennen und zu entscheiden welches Handeln in einer bestimmten Situation angemessen ist \cite{dey_understanding_2001}. Der limitierende Faktor des Potenzials eines kontextsensitiven Systems ist das Menge an Kontext die ein Entwickler vorhersehen und kodieren kann.
%The list of unforeseen or undetectable contexts can be endless.Summarizing, if developers can not determine all that can be affected by an action, it will be very difficult to write a closed and comprehensive set of actions to take in those cases.
Es ist aber weder beim Design noch später bei der Implementierung unmöglich alle Zusammenhänge vorherzusehen. Dementsprechend schwer wird ein in sich geschlossenes und allumfassendes Regelset festzulegen. \cite{perera_context_2014}   

%“(A) Enumerate the set of contextual states that may exist;” 
%“(B) Know what information could accurately determine a contextual state within that set;” 
%“(C) State what appropriate action should be taken in a particular state.”
%TODO raussuchen: S. Greenberg, Context as a dynamic construct, Human-Computer Interaction 16 (2) (2001) 257–268. 
Nach \cite{TODO} gibt es 3 Aufgaben die man sich beim Entwerfen eines kontextsensitiven Systems zu lösen versuchen sollte: 
Spezifizieren aller möglichen Kontextzustände.
Wissen welche Informationen einen konkreten Kontextzustand akkurat festlegen.
Welche Aktion im jeweiligen Zustand ausgeführt werden sollen.
		\subsection{ Festlegung auf bestimmte IDS-Implementierungen}
		\subsection{ Analyse der IDS-Regel-Syntax}
		\subsection{ Abgleich zwischen Taxonomie und den in der Realität zur Verfügung stehenden Informationen}
		
