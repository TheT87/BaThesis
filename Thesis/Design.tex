\chapter{Design}%
\label{cha:design}

In this chapter, you should present your solution in detail but at a conceptual level. This means that you explain the overall design including your motivation for this design but you do not provide details on the actual implementation of your design (e.g., in which programming language you wrote it, how the software is structured and so on). This means that you should also point out the aspects where you had different design options and in which points they differ. A good approach to write this chapter is to make yourself aware of the different aspects and design problems that need to be addressed in your design. To do so, you can then proceed repeatedly in three steps:
\begin{enumerate}
   \item Explain a design problem that needs to addressed by the solution (e.g. to enable anonymous communication over the internet, participants need to be able to send messages to each other without revealing identifying to the corresponding receiver).
   \item Discussion of design choices (e.g. Mix Networks, DC-Networks, etc.) with regards to the requirements from the previous chapter and identification of the most promising choices.
\end{enumerate}
After the second step, you start the next iteration by identifying design problems that arise when you want to use the most promising design choice. For example, if Mix networks turn out to be the most promising approach for your requirements, you then need to address the question how the mix network should be designed (e.g. how are mix nodes chosen by the users of the anonymization network? How do mix nodes process messages?). Once you identified the most promising solutions to that, you can then start the next iteration and so on until there are no more open design questions that you are aware of.

\section{“Übersetzung”/Transfer der Taxonomie/Kategorien in IDS-Signaturen} 

		\subsection{ Festlegung auf bestimmte IDS-Implementierungen}
		\subsection{ Analyse der IDS-Regel-Syntax  }
		\subsection{ Abgleich zwischen Taxonomie und den in der Realität zur Verfügung stehenden Informationen}
		\subsection{ Umsetzung/Implementierung der Taxonomie in  IDS-Signaturen}
