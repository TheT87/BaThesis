\chapter{Einführung}
\label{cha:introduction}

%The introduction serves the purpose of educating the reader about the broader context or research problem that your work addresses and its relevance as well as the relevance of your work towards solving this problem.
%As a rough guideline, this section should be written to answer the following questions:
%\begin{enumerate}
%   \item What is the problem your work addresses?
%   \item Why is it desirable to solve this problem? (e.g. which new possibilities/use cases become possible if the problem is solved)
 %  \item Roughly, what contribution towards solving the problem will you present to the reader in this thesis?
%\end{enumerate}

Seit Computernetzwerke existieren, gibt es auch Menschen, welche sich zu diesen widerrechtlichen Zugang verschaffen wollen. Weder Sicherheitsmaßnahmen noch deren Weiterentwicklung haben diesen Umstand bisher geändert. Lediglich die Netzwerke, welche dabei von Interesse sind, haben sich gewandelt und stellen die IT-Sicherheit dabei immer wieder vor neue Herausforderungen, die zunehmend nicht mehr ausschließlich manuell gelöst werden können. Gleichzeitig verändern sich auch Angreifer und ihre Angriffsarten und die dabei verwendeten Werkzeuge und Techniken entwickeln sich weiter und ermöglichen nach wie vor, in kontrollierte Umgebungen einzudringen, Schäden zu verursachen und dabei trotzdem unentdeckt zu bleiben. Dagegen stehen einem Netzwerkadministrator allerdings eine Vielzahl von Mitteln zur Verfügung. Unter anderem Netzwerk-Sicherheits-Monitoring, dessen erste Stufe beim Umgang mit Angriffen die Erkennung darstellt. Etabliert hat sich dabei vor allem die Verwendung von Intrusion Detection Systemen, welche aber wie jedes System dieses Komplexitätsgrades nie gänzlich fehlerfrei arbeiten. Deshalb ist es letztendlich immer noch nötig, dass die Entscheidungen eines IDS regelmäßig durch einen Menschen überprüft und gegebenenfalls korrigiert werden. Um also die Menge an Meldungen, welche manuell überprüft werden müssen, zu verringern und die Aussagekraft der erzeugten Alarme zu erhöhen und damit den menschlichen Entscheider zu entlasten sowie generell seine Aufgaben besser zu erfüllen, muss ein IDS eine möglichst große Zahl von Szenarien korrekt einschätzen. Dazu werden Informationen benötigt, mit denen die Geschehnisse in einem Netzwerk in einer für das IDS verständlichen Form vollständig abgebildet werden können. In dieser Arbeit soll ergründet werden, auf welche Art welche Informationen automatisch gesammelt 
und verwertet werden können, um eine Verbesserung der Leistung unter Einbeziehung von zusätzlichem Kontext in die Entscheidungsfindung kontextsensitiver Systeme zu erreichen und welche Kategorien von Kontextinformationen dafür am besten geeignet sind.