\chapter{Einführung}
\label{cha:introduction}

%The introduction serves the purpose of educating the reader about the broader context or research problem that your work addresses and its relevance as well as the relevance of your work towards solving this problem.
%As a rough guideline, this section should be written to answer the following questions:
%\begin{enumerate}
%   \item What is the problem your work addresses?
Seit es in der Geschichte der Menschheit abgeschlossene Räume gibt, existieren auch Menschen die sich widerrechtlich zu diesen Zugang verschaffen wollen. Weder der fortschreitende Grad an Zivilisation noch die zunehmende Technologisierung haben daran etwas geändert. Lediglich die Räume in welche dabei von Interesse sind haben sich gewandelt. Heute sind Computernetzwerke das neue Ziel der Wahl. Mit dem technologischen Fortschritt entwickeln sich auch die Angreifer ihre Angriffsarten sowie die dazu verwendeten Werkzeuge und Techniken, die es ermöglichen, in komplexere oder besser kontrollierte Umgebungen einzudringen und größere Schäden zu verursachen und dabei zunehmend unerkannt zu bleiben weiter.
Dagegen stehen dem Netzwerkadministrator von Welt eine Vielzahl von Mitteln zur Verfügung. Unter anderem Netzwerk-Sicherheits-Monitoring, dessen erster Aspekt die Erkennung solcher Angriffe darstellt.
%   \item Why is it desirable to solve this problem? (e.g. which new possibilities/use cases become possible if the problem is solved)
Die Erkennungskomponenten von NSM zu denen IDS ebenfalls zählen, arbeiten allerdings quasi nie komplett fehlerfrei. Deshalb wird letztendlich immer noch ein Mensch benötigt um die Entscheidungen des IDS regelmäßig zu überprüfen und gegebenenfalls zu korrigieren. Um also die Aussagekraft zu erhöhen und so die Menge an Meldungen welche manuell überprüft werden müssen zu verringern sowie generell seine Aufgaben zu erfüllen, muss ein IDS eine möglichst große Zahl von Szenarien korrekt einschätzen. Dazu werden Informationen benötigt mit denen die Geschehnisse in einem Netzwerk in einer für das IDS verständlichen Form vollständig abgebildet werden können.      
 %  \item Roughly, what contribution towards solving the problem will you present to the reader in this thesis?

%\end{enumerate}

%\section{Mängel bestehender IDS}
%\subsection{Festlegung der IDS-Performancekriterien}
%\section{Verbesserung bestehender IDS-Lösungen(bessere/weniger Zugriffskontrollentscheidungen)}   
%\section{Erläuterung der Vorteile eines kontextsensitiven IDS}
%\section{}
%- Es gibt Dinge auf die nicht jede Entität zugreifen darf
%- Detection, prevention, 