\chapter{Implementierung}%
\label{cha:implementation}

%In this chapter, you should provide technical details on how you actually implemented the design that you derived in the previous chapter.

% Comparative study and analysis of network intrusion detection tools

		\subsection{ Umsetzung/Implementierung der Taxonomie in  IDS-Signaturen}
\section{Test der kontextsensitiven Signaturen}

\subsection{ (Auswahl eines bereits existierenden/ Erstellung eines eigenen) Datensatzes + dazugehörige Label }
\subsection{ Aufbau eines oder mehrerer Netzwerke  } 
\subsection{ Setup der verschiedenen IDS }
\subsection{ Grundlage mit non-kontextsensitiven Signaturen auf Datensatz }
\subsection{ Test der kontextsensitiven Signaturen auf Datensatz} 

\section{Netzwerk}
%Das Netzwerk besteht aus einem Sys


\section{Netzwerkverkehr}
Der Netzwerkverkehr wurde mittels Scapy erzeugt. Das erlaubt die Kontrolle und schnelle Anpassung der verschiedenen Schichten eines Packets.
\subsection{Geografische Koordinaten und Ortszeit}
\subsection{Größe des Netzwerks}
\subsection{Rechte eines Nutzers}
\section{Angriffe}
\subsection{ARP-Spoofing}
\subsection{NAT-Slipstreaming}
\subsection{DoS/DDoS}
\subsection{}