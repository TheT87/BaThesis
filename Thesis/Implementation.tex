\chapter{Implementierung}%
\label{cha:implementation}

%In this chapter, you should provide technical details on how you actually implemented the design that you derived in the previous chapter.

% Comparative study and analysis of network intrusion detection tools

		\subsection{ Umsetzung/Implementierung der Taxonomie in  IDS-Signaturen}
\section{Test der kontextsensitiven Signaturen}

\subsection{ (Auswahl eines bereits existierenden/ Erstellung eines eigenen) Datensatzes + dazugehörige Label }
\subsection{ Aufbau eines oder mehrerer Netzwerke  } 
\subsection{ Setup der verschiedenen IDS }
\subsection{ Grundlage mit non-kontextsensitiven Signaturen auf Datensatz }
\subsection{ Test der kontextsensitiven Signaturen auf Datensatz} 

\section{Netzwerk}
Das Netzwerk besteht aus vier Docker Containern. 

\begin{TAB}(r,1cm,1cm)[1pt]{|c|c|c|}{|c|c|c|c|c|}% (rows,min,max)[tabcolsep]{columns}{rows}
Name & Zweck & IP-Adresse + Port \\ 
Teinehmer 1 	+ Opfer	& Für die Relasisierung verschiedener Angriffe & 172.17.0.5 + 40000 \\ 
Teilnehmer 2 + Angreifer & Für die Relasisierung verschiedener Angriffe & 172.17.0.4 + 40001 \\
Angreifer & Container der mithilfe von Scapy Netzwerkverkehr simuliert & 172.17.0.3 \\
Wireshark + Zeek + osquery & Mitschnitt, Speicherung, Verarbeitung und Auswertung des Netzwerkverkehrs & 172.26.144.132 \\
\end{TAB}



\section{Netzwerkverkehr}
\section{Angriffe}
\subsection{ARP-Spoofing}
\subsection{NAT-Slipstreaming}
\subsection{DoS/DDoS}
\subsection{}